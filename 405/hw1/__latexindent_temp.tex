\documentclass{article}
    \usepackage[margin=1in]{geometry}
    \usepackage{amsmath,cancel}
    \usepackage{graphicx}
    \usepackage{alltt}
  
  \title{CSCI 405, Homework \# 1}
  \author{Caleb Ouellette}
  
  
  \begin{document}
  
  \maketitle
  \begin{enumerate}
  \item
    \begin{align*}
      n^2 - 16n &= \Theta(n^2) \\
      0 < c_1 n^2 < n^2 - 16n &< c_2 n^2 < \infty
    \end{align*}
    Finding $c_1$
    \begin{align*}
      c_1 n^2 &< n^2 - 16n \\
      c_1 n^2 &< n(n - 16)
    \end{align*}
    let $n_0$ be 17.
    \begin{align*}
      c_1 17^2 &< 17(17 - 16) \\
      c_1 17 &< 1 \\
      c_1  &< \frac{1}{17}
    \end{align*}
    Finding $c_2$ is trivial and can be satisifyed by just using 1.\\
    $c_1 = \frac{1}{17}$\\
    $c_2 = 1$\\
    $n_0 = 17$\\

  
  \item 
    \begin{align*}
      T(n)= 2T(\frac{n}{2}) + n\\
    \end{align*}
    $T(n) = O(n)$ implies $T(n) \leq cn $. Thus we can subsitute $T(n)$ for $cn$ letting c be some constant.
    \begin{align*}
      cn &= 2c(\frac{n}{2}) + n\\
      cn &= cn + n\\
      cn &= n(c + 1)\\
      cn &\neq n(c + 1)\\
    \end{align*}
    There is no value for c that makes this true. Thus the Assumption that $T(n) = O(n)$ is false.

  \item
    \begin{align*}
      f(0) &= 7 \\
      f(n) &= 2f(n - 1) \\
    \end{align*}
    a.
    \begin{align*}
    f(n) &= \cancel{2f(n-1)}\\
    \cancel{2f(n-1)} &= \cancel{2^2f(n-2)}\\
    \cancel{2^2f(n-2)} &= \cancel{2^3f(n-3)}\\
    \cancel{2^3f(n-3)} &= \cancel{2^4f(n-4)}\\
    &\ldots\\
    \cancel{2^{k-1}f(n-(k-1)} &= 2^{k}f(n-k)
  \end{align*}
  Consider n = k being the last case
  \begin{align*}
    f(n) &= 2^{n}f(0) \\
    f(n) &= (7)2^{n}
  \end{align*}
  b.
  \begin{align*}
    f(n) &= 2f(n - 1) \\
    (7)2^{n} &= 2((7)2^{n - 1}) \\
    (7)2^{n} &= 2(1/2(7)2^{n}) \\
    (7)2^{n} &= (7)2^{n} \\
  \end{align*}

  \item
  $3+7+11+ ... +283$

   \begin{align*}
    \sum_{i=0}^{n}(4i + 3) \\
    \sum_{i=0}^{n}4i + \sum_{i=0}^{n}3 \\
    4\sum_{i=0}^{n}i + 3(n + 1)\\
    2(n + 1)n + 3(n + 1) \\
    2n^2 + 2n + 3n + 3 \\
    2n^2 + 5n + 3 \\
  \end{align*}
  (283 - 3 )/ 4 = 70 \\
  \begin{align*}
    2(70)^2 + 5(70) + 3 \\
    9800 + 350 + 3 \\
    10153 \\
  \end{align*}
  
  
  \item 
  $(1+2+...+n)+(2+3+...+n)+(3+4+...n)+... +n$
   \begin{align*}
    \sum_{i=1}^{n}i^i \\
    \sum_{i=1}^{n}\sum_{j=i}^{n}j \\
    \sum_{i=1}^{n}\sum_{j=1}^{n}j - \sum_{x=1}^{i - 1}x  \\
    \end{align*}
  \item 
    \begin{align*}
    \sum_{i=1}^{n}\sum_{j=1}^{n}j - \sum_{x=1}^{i}x + i  \\
    \sum_{i=1}^{n} \frac{n(n+1)}{2} - \frac{i(i+1)}{2} + i  \\
    \sum_{i=1}^{n} \frac{n(n+1)}{2} - \sum_{i=1}^{n} \frac{i(i-1)}{2} + \sum_{i=1}^{n}i  \\
    \frac{n^2(n+1)}{2} - \sum_{i=1}^{n} \frac{i(i+1)}{2} + \frac{n(n+1)}{2}  \\
    \frac{n^2(n+1)}{2} - 1/2(\sum_{i=1}^{n} i^2+i) + \frac{n(n+1)}{2}\\
    \frac{n^2(n+1)}{2} - 1/2(\frac{n(n+1)(2n +1)}{6}- \frac{n(n+1)}{2}) + \frac{n(n+1)}{2} 
   \end{align*}

  

  \end{enumerate}
  \end{document}