\documentclass{article}
    \usepackage[margin=1in]{geometry}
    \usepackage{amsmath,cancel}
    \usepackage{graphicx}
    \usepackage{alltt}
  
  \title{CSCI 305, Homework \# 1}
  \author{Caleb Ouellette}
  \date{Due date: Tue, April 24, Midnight}
  
  
  \begin{document}
  
  \maketitle
  
  Find solutions to each of the following recurrences
  using the techniques outlined in the lecture notes (``without guessing'').
  Demonstrate that your closed form finds the same values for several
  small numbers, and prove by induction that your closed form satisfies
  the recurrence in general.
  
  \begin{enumerate}
  \item
    \begin{align*}
      f(0) &= 3\\
      f(n) &= 5f(n-1)
    \end{align*}
    Sequence:
    3, 15, 75...
  
    Solution:
    \begin{align*}
      f(n) &= \cancel{5f(n-1)}\\
      \cancel{5f(n-1)} &= \cancel{5^2f(n-2)}\\
      \cancel{5^2f(n-2)} &= \cancel{5^3f(n-3)}\\
      \cancel{5^3f(n-3)} &= \cancel{5^4f(n-4)}\\
      &\ldots\\
      \cancel{5^{k-1}f(n-(k-1)} &= 5^{k}f(n-k)
    \end{align*}
      Consider n = k being the last case
    \begin{align*}
      f(n) &= 5^{n}f(0)
    \end{align*}
    Using the fact that $f(0)=3$ gives 
    \begin{align*}
      f(n) &= 5^{n}3
    \end{align*}
  
    Small number checks
    \begin{align*}
      f(0) &= 5^{0}3 = 3\\
      f(1) &= 5^{1}3 = 15\\
      f(2) &= 5^{2}3 = 75\\
    \end{align*}
  
  
    Prove
    \begin{align*}
      f(n + 1) &= 5^{n + 1}3 \\
    \end{align*}    
    Proof
    \begin{align*}
      f(n + 1) &= 5f(n) \\    
      &= 5(5^n3) \\
      &= 5^{n + 1}3 \\
    \end{align*}
  
  
  
  \end{enumerate}
  
  \end{document}