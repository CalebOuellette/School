\documentclass{article}
\usepackage[margin=1in]{geometry}
\usepackage{amsmath}
\usepackage{clrscode3e}

\title{CSCI 305, Homework \# 7}
\author{Caleb Ouellette}
\date{Due date:  Midnight, Tuesday, May 29}


\begin{document}

\maketitle

\begin{description}
\item[Quadratic probing.]  This is problem 11-3
  in the book.

  Suppose that we are given a key $k$ to search for in a hash table
  with positions $0,1,...,m-1$, and suppose that we have a hash
  function $h$ mapping the key space into the set $\{0,1,...,m-1\}$.
  The search scheme is as follows:

  \begin{enumerate}
  \item Compute the value $j=h(k)$ and set $i=0$.
    \item Probe in position $j$ for the desired key $k$.  If you find
      it, or if this position is empty, terminate the search.
      \item Set $i=i+1$.  If $i$ now equals $m$, the table is full, so
        terminate the search.  Otherwise, set $j=(i+j)\mod m$ and
        return to step 2.
  \end{enumerate}
  Assume that $m$ is a power of 2.
  \renewcommand{\theenumi}{\alph{enumi}}
  \begin{enumerate}
    \item Show that this scheme is an instance of the general
      ``quadratic probing'' scheme by exhibiting the appropriate
      constants $c_1$ and $c_2$ for equation (11.5). \\

      The equation above can be written as \\
      \begin{align*}
      h(k, 0) &= j\\
      h(k, 1) &= j + 1 \mod m\\
      h(k, 2) &= j + 1 + 2 \mod m\\
      h(k, 3) &= j + 1 + 2 + 3 \mod m\\
      h(k, i) &= j + \sum_{q=1}^i q  \mod m\\
      h(k, i) &= j + \frac{i(i + 1)}{2}  \mod m\\
      h(k, i) &= j + \frac{i^2}{2} + \frac{i}{2}  \mod m\\
      \end{align*}

      The general form of the equation is \\
      \begin{align*}
      h(k,i) &= (h(k) + c_1 i + c_2i^2) \mod m\\
      \end{align*}
      Choose $c_1 = 1/2$ and $c_2 = 1/2$ \\
      \begin{align*}
      h(k,i) &= (h(k) + i/2 + i^2/2) \mod m \\
      h(k,i) &= j + \frac{i}{2} + \frac{i^2}{2}  \mod m \\
      \end{align*}
      \item Prove that this algorithm examines every table position in
        the worst case.\\

        There are 3 ways to end a search. 1. We find the key. 2. We find an empty entry. 3. i = m.
        In the worst case we don't find the key and we have a full table. So case 1 and 2 are never met. 
        Leaving only case 3. Case 3 guarentees that we will not look at more than m entries because we stop
        when i = m. \\

        Now to prove this claim we must show that over the course of those m tries we don't look at the same item twice.
        In order to visit the same node twice we would have to get two unqiue numbers such that h(k, i) = h(k, q) where both i and q are less then m-1 and greater then 0.

         \begin{align*}
      j + \frac{i}{2} + \frac{i^2}{2}  \mod m &= j + \frac{q}{2} + \frac{q^2}{2}  \mod m \\
      \frac{i}{2} + \frac{i^2}{2}  \mod m &= \frac{q}{2} + \frac{q^2}{2}  \mod m \\
      \frac{i}{2} + \frac{i^2}{2} - (\frac{q}{2} + \frac{q^2}{2}) \mod m &=  0 \\
      \frac{(i + i^2) - (q + q^2)}{2}) \mod m &=  0 \\
      \frac{i(1 + i) - q(1 - q)}{2} \mod m &=  0 \\
      \frac{(i- q) (i + q + 1)}{2} \mod m &=  0 \\
      \end{align*}
      To get mod m to equal 0, (i - q) must equal m, (i + q + 1) must equal m or (i - q)(i + q + 1) must equal m. \\

      Both i and q are less then m, so i - q is diffently not m. \\

      (i + q + 1) could equal m, but if it does then (i - q) is forced to be odd. Because of the divide by two can't be taken out of the odd, it will force (i + q + 1) to be equal to m/2 , not m. \\

      (i + q + 1) can't equal 2m because of i and q are less then m -1. \\

      (i - q)(i + q + 1) can't equal m because one will be odd and the other even, so can't make a power of two.


  \end{enumerate}
  

\end{description}


\end{document}