\documentclass{article}
  \usepackage[margin=1in]{geometry}
  \usepackage{amsmath,cancel}
  \usepackage{graphicx}
  \usepackage{alltt}

\title{CSCI 305, Homework \# 1}
\author{YOUR NAME HERE}
\date{Due date: Tue, April 24, Midnight}


\begin{document}

\maketitle

Find solutions to each of the following recurrences
using the techniques outlined in the lecture notes (``without guessing'').
Demonstrate that your closed form finds the same values for several
small numbers, and prove by induction that your closed form satisfies
the recurrence in general.

\begin{enumerate}
\item
  \begin{align*}
    f(0) &= 3\\
    f(n) &= 5f(n-1)
  \end{align*}
  Solution:
  \begin{align*}
    f(n) &= \cancel{5f(n-1)}\\
    \cancel{5f(n-1)} &= \cancel{5^2f(n-2)}\\
    \cancel{5^2f(n-2)} &= \cancel{5^3f(n-3)}\\
    \cancel{5^3f(n-3)} &= \cancel{5^4f(n-4)}\\
    &\ldots\\
    \cancel{5^{k-1}f(n-(k-1)} &= 5^{k}f(n-k)
  \end{align*}
    Consider n = k being the last case
  \begin{align*}
    f(n) &= 5^{n}f(0)
  \end{align*}
  Using the fact that $f(0)=3$ gives 
  \begin{align*}
    f(n) &= 5^{n}3
  \end{align*}
  Prove
  \begin{align*}
    f(n + 1) &= 5^{n + 1}3 \\
  \end{align*}    
  Proof
  \begin{align*}
    f(n + 1) &= 5f(n) \\    
    &= 5(5^n3) \\
    &= 5^{n + 1}3 \\
  \end{align*}


\item
  \begin{align*}
    f(0) &= 5\\
    f(n) &= f(n-1) + 4
  \end{align*}
  Solution:
  \begin{align*}
    f(n) &= \cancel{f(n-1)}+4\\ 
    \cancel{f(n-1)} &= \cancel{f(n-2)} +4\\ 
    \cancel{f(n-2)} &= \cancel{f(n-3)} +4\\
    \cancel{f(n-3)} &= \cancel{f(n-4)} +4\\
    &\ldots\\
    \cancel{f(n-(k-1))} &= f(n-k)+4
  \end{align*}
  Consider n = k being the last case
  \begin{align*}
    f(n) &= f(0)+ \sum_{i=1}^{n}4 \\
    f(n) &= 5 + 4n \\
  \end{align*}
  Prove
  \begin{align*}
    f(n+1) &= 5 + 4(n + 1)
  \end{align*}
  Proof
  \begin{align*}
    f(n+1) &= f(n) + 4 \\
    &=  5 + 4n + 4 \\
    &=  5 + 4(n + 1)\\
  \end{align*}

\item
  \begin{align*}
    f(0) &= 2\\
    f(n) &= 4f(n-1) + 6
  \end{align*}
  Solution:
  \begin{align*}
    f(n) &= \cancel{4f(n-1)}+6\\ 
    \cancel{4f(n-1)} &= \cancel{4^2f(n-2)} +4(6)\\ 
    \cancel{4^2f(n-2)} &= \cancel{4^3f(n-3)} +4^2(6)\\ 
    \cancel{4^3f(n-3)} &= \cancel{4^4f(n-4)} +4^3(6)\\ 
    &\ldots\\
    \cancel{4^{k-1}f(n-(k-1))} &= 4^kf(n-k)+4^{k-1}6
  \end{align*}
  Consider n = k being the last case
  \begin{align*}
    f(n) &= 4^nf(0)+ \sum_{i=0}^{n-1}4^i6 \\
    f(n) &= 4^n(2)+ 6\sum_{i=0}^{n}4^i - 4^n(6) \\
    f(n) &= 4^n(2)+ 6\frac{4^{n+1} - 1}{4-1} - 4^n(6) \\
    f(n) &= 4^n(2) - 4^n(6) + 6\frac{4^{n+1} - 1}{3}  \\
    f(n) &= -4^{n + 1} + 2(4^{n+1} - 1)  \\
    f(n) &= -4^{n + 1} + (2)4^{n+1} - 2  \\
    f(n) &= 4^{n+1} - 2  \\
  \end{align*}
  Prove
  \begin{align*}
  \end{align*}


\item
  \begin{align*}
    f(0) &= 3\\
    f(n) &= 5f(n-1) + n
  \end{align*}
  Solution:
  \begin{align*}
    f(n) &= \cancel{5f(n-1)}+n\\ 
    \cancel{5f(n-1)} &= \cancel{5^2f(n-2)} +5(n-1)\\
    \cancel{5^2f(n-2)} &= \cancel{5^3f(n-3)} +5^2(n-2)\\
    \cancel{5^3f(n-3)} &= \cancel{5^4f(n-4)} +5^3(n-3)\\
    &\ldots\\
    \cancel{5^{k-1} f(n-(k-1))} &= {5^kf(n-k)} +5^{k-1}(n-(k-1))\\
  \end{align*}
  Consider n = k being the last case
  \begin{align*}
    f(n) &= {5^nf(0)} + \sum_{i=0}^{n-1}5^i(n-i) \\
    f(n) &= {5^n3} + \sum_{i=0}^{n-1}n5^i - \sum_{i=0}^{n-1}i5^i \\
    f(n) &= {5^n3} + n\sum_{i=0}^{n-1}5^i - (\sum_{i=0}^{n}i5^i -n5^n) \\
    f(n) &= {5^n3} + n(\frac{1-5^n}{ 1 - 5}) - (\frac{5 - (n +1)5^{n+1} +n5^{n+2}}{ (5 - 1)^2})  + n5^n \\
    f(n) &= {5^n3} + n(\frac{1-5^n}{ 4}) - (\frac{5 - (n +1)5^{n+1} +n5^{n+2}}{ 16})  + n5^n \\
  \end{align*}

\end{enumerate}

\end{document}