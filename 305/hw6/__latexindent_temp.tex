\documentclass{article}
  \usepackage[margin=1in]{geometry}
  \usepackage{amsmath}
  
  \title{CSCI 305, Homework \# 6}
  \author{YOUR NAME HERE}
  \date{Due date:  Midnight, May 14}
  
  
  \begin{document}
  
  \maketitle
  
  \begin{enumerate}
  \item Alternative quicksort analysis (problem 7-3 in the text).
  
    An alternative analysis of the running time of randomized quicksort
    focuses on the expected running time of each individial recursive
    call to {\sc Randomized-Quicksort}, rather than on the number of
    comparisons performed.
  
    \begin{enumerate}
      \item Argue that, given an array of size $n$, the probability that
        any particular element is chosen as the pivot is $1/n$.  Use
        this to define indicator random variables
        \[
        X_i = I\{\mbox{$i$th smallest element is chosen as the pivot}\}
        \]
        What is $E[X_i]$?
  
      \item
        Let $T(n)$ be a random variable denoting the running time of
        quicksort on an array of size $n$.  Argue that
        \begin{align}
        E[T(n)] &= E\left[
          \sum_{q=1}^n X_q(T(q-1) + T(n-q) + \Theta(n))
          \right]
        \label{eqn1}
        \end{align}
  
      \item
        Show that we can rewrite equation \ref{eqn1} as
        \begin{align}
          E[T(n)] &= \frac{2}{n}\sum_{q=2}^{n-1}E[T(q)] + \Theta(n)
            \label{eqn2}
        \end{align}
  
      \item
        Show that
        \begin{align}
          \sum_{k=2}^{n-1} k \lg k \leq \frac{1}{2}n^2\lg n -
          \frac{1}{8} n^2
          \label{eqn3}
        \end{align}
        (Hint: Split the summation into two parts, one for
        $k=2,3,...,\lceil n/2 \rceil - 1$
        and one for
        $k= \lceil n/2\rceil,...,n-1$
  
      \item
        Using the bound from equation \ref{eqn3}, show that the
        recurrence in equation \ref{eqn2} has the solution
        $E[T(n)] = \Theta(n\lg n)$.  (Hint: Show, by substitution, that
        $E[T(n)\leq an\lg n$ for sufficiently large $n$ and for some
          positive constant $a$.) 
  
    \end{enumerate}
  \end{enumerate}
  
  \end{document}
  