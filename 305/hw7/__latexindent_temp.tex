\documentclass{article}
\usepackage[margin=1in]{geometry}
\usepackage{amsmath}
\usepackage{clrscode3e}

\title{CSCI 305, Homework \# 7}
\author{Caleb Ouellette}
\date{Due date:  Midnight, Tuesday, May 29}


\begin{document}

\maketitle

\begin{description}
\item[Quadratic probing.]  This is problem 11-3
  in the book.

  Suppose that we are given a key $k$ to search for in a hash table
  with positions $0,1,...,m-1$, and suppose that we have a hash
  function $h$ mapping the key space into the set $\{0,1,...,m-1\}$.
  The search scheme is as follows:

  \begin{enumerate}
  \item Compute the value $j=h(k)$ and set $i=0$.
    \item Probe in position $j$ for the desired key $k$.  If you find
      it, or if this position is empty, terminate the search.
      \item Set $i=i+1$.  If $i$ now equals $m$, the table is full, so
        terminate the search.  Otherwise, set $j=(i+j)\mod m$ and
        return to step 2.
  \end{enumerate}
  Assume that $m$ is a power of 2.
  \renewcommand{\theenumi}{\alph{enumi}}
  \begin{enumerate}
    \item Show that this scheme is an instance of the general
      ``quadratic probing'' scheme by exhibiting the appropriate
      constants $c_1$ and $c_2$ for equation (11.5).
      \item Prove that this algorithm examines every table position in
        the worst case.
  \end{enumerate}
  

\end{description}


\end{document}