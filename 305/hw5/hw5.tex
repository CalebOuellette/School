\documentclass{article}
  \usepackage[margin=1in]{geometry}
  \usepackage{amsmath}
  \usepackage{clrscode3e}
  \usepackage{mathtools}
  \DeclarePairedDelimiter\ceil{\lceil}{\rceil}
  \DeclarePairedDelimiter\floor{\lfloor}{\rfloor}
  \title{CSCI 305, Homework \# 5}
  \author{Caleb Ouellette}
  \date{Due date:  Midnight, May 14}
  
  
  \begin{document}
  
  \maketitle
  
  \begin{enumerate}
  \item Analysis of d-ary heaps (problem 6-2 in the text).
  
  A \textbf{d-ary heap} is like a binary heap, but (with one possible
  exception) non-leaf nodes have $d$ children instead of 2 children.
  \begin{enumerate}
  \item How would you represent a $d$-ary heap in an array? \\
    Use the same strucutre as a binary tree, exect use h instead of 2. So the root is still at one, 
    then its h children. After that is the h children on the left child of the first node and so on.
    
  \item What is the height of a $d$-ary heap of $n$ elements in terms
    of $n$ and $d$? \\
    \begin{equation*}
    \floor*{log_h n}
    \end{equation*}

  \item Give an efficient implementation of \textsc{Extract-Max} in
      a $d$-ary max-heap.  Analyze its running time in terms of $d$
    and $n$.
        \begin{codebox}
          \Procname{$\proc{Extract-Max}(A)$}
          \li \If $n < 1$ \Do
          \li \Error ``heap underflow''
        \End
        \li $max \gets A[1]$
        \li $A[1] = A[n]$
        \li $n = n-1$
        \li $\proc{Max-Heapify}(A,1,n)$
        \li \Return $max$
        \end{codebox}
        \ The Extract-Max function will be dominated by Max-Heapify. 
        Max-Heapify will do $\floor*{log_h n}$ swaps to move the top 
        element to the bottom of the tree and take $O(log_h n)$ time.
        
      \item Give an efficient implementation of \textsc{Insert} in
        a $d$-ary max-heap.  Analyze its running time in terms of $d$
        and $n$.

        \begin{codebox}
          \Procname{$\proc{Insert}(A,key,n)$}
          \li $n \gets n+1$
          \li $A[n] \gets -\infty$
          \li $\proc{Increase-Key}(A,n,key)$
        \end{codebox}

        Heap-Increase-Key will move a bottom element to at most the top of the 
        tree, so it will take $O(log_h n)$ time.

      \item Give an efficient implementation of
        \textsc{Increase-Key}$(A,i,k)$,
        which flags an error if $k < A[i]$, but otherwise sets $A[i] =
        k$ and then updates the $d$-ary max-heap structure
        appropriately.    Analyze its running time in terms of $d$
        and $n$.
        \begin{codebox}
          \Procname{$\proc{Increase-Key}(A,i,key)$}
          \li \If $key < A[i]$ \Do
          \li \Error ``new key is smaller than current key''
        \End
        \li $A[i] = key$
        \li \While $i > 1$ and $A[\proc{Parent}(i)] < A[i]$ \Do
        \li exchange $A[i]$ with $A[\proc{Parent}(i)]$
        \li $i = \proc{Parent}(i)$
        \End
        \end{codebox}

   \end{enumerate}
  \end{enumerate}
  
  \end{document}
  